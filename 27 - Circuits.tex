\documentclass[]{article}
\usepackage[T1]{fontenc}
\usepackage{siunitx}
\usepackage{amsmath}
\usepackage{parskip}
\usepackage{physics}

\begin{document}
\section{Resistance}
Resistance in series:
\[ R_{\text{eq}} = \sum_{k} R_k \]
Resistance in parallel:
\[ \frac{1}{R_{\text{eq}}} = \sum_{k} \frac{1}{R_k} \]
It's basically the reverse of capacitance.

\section{emf}
Every circuit needs \emph{some} source of electric energy. Usually, this is accomplished by a battery that maintains a constant potential difference $ V $. Let's think about it slightly differently:
\[ \mathcal{E} = \dv{W}{q} \]
We call this value \emph{EMF}. Here, $ \text{d}W $ is the amount of work the emf device does internally on a charge $ \text{d}q $.

The difference between EMF and voltage is that the EMF is the potential difference measured across a power source without a load connected to it whereas voltage is the potential difference measured between any two points in a circuit. We use the term EMF when we speak about batteries, generators, transformers, and other power sources whereas the term voltage is used when we speak about loads, circuits, and circuit components.

\section{Thinking in Circuits}
This kinda sucks without visualization, but I'm too lazy to get a circuit package so we'll have to make do without them. Here's some key things to note:
\begin{itemize}
  \item In a battery-resistor circuit, current is \emph{only} split when a wire is in parallel. It never changes across individual resistors in series.
  \item Electric Potential is what drops across circuits (covered in the next section)
\end{itemize}

\subsection{Single-Loop Simple Circuit}
In a simple battery-resistor circuit between two points, we can think of the potential difference as we run along as a graph of \emph{position vs potential}. For the picture presentation of this, see page 776. The battery increases emf from 0 to the value $ \mathcal{E} $ of the battery. From there, each individual resistor decreases the potential by $ iR $ until it reaches zero. Essentially,
\[ \mathcal{E} - \sum_{k} iR_k = 0 \]
The \emph{electric potential} varies as current runs along the resistors. This phenomenon is called \emph{voltage drop}.

\subsubsection{Multiple Batteries}
Treat the emf generated by a battery as a direct addition/subtraction of emf depending on the direction it is in.

\subsection{Multiple Batteries (Double Loop)}
There are circuits with two batteries that are \emph{not} able to be reduced or thought of as a single-battery circuit.

One particular circuit we looked at involved two batteries on two sides, ground under both of them, and two loops on both sides with resistors in each gap. The middle resistor will have a current that is the sum of the top two currents. From there, solve each loop individually.

\section{Ammeter and Voltmeter}
An ideal ammeter is in series and has zero internal resistance, while an ideal voltmeter is in parallel and has infinite internal resistance.

\section{RC Circuits}
An RC circuit consists only of a resistor and capacitor, along with an emf source to charge the capacitor at the very start. As learned before, a capacitor doesn't store emf directly, but rather a charge, and the electric field produced from it is what continues the electric potential.

\subsection{Charging a Capacitor}
The intricacies on how to derive the equation for charging a capacitor are not relevant for this course. If you want to know more, see page 789. All you need to know is that when charging a capacitor, the charge changes with respect to time as follows:
\[ q = C \mathcal{E} \left(1 - e^{-t/\left(RC\right)}\right) \]
Note that $ e $ here is not the elementary charge, but the mathematical constant.

Using the notions that $ i = \dv{q}{t} $ and $ q = CV $, we can derive the following two equations as well:
\[ i = \left(\frac{\mathcal{E}}{R}\right) e^{-t/(RC)} \]
\[ V_c = \mathcal{E}\left(1 - e^{-t/(RC)}\right) \]

We know that $ R $, $ C $, and $ E $ are constant in an RC circuit, so the only variable that matters is $ t $. For this course, the in-between does not really matter. What matters is how these equations behave at $ t = 0 $ and as $ t \to \infty $.

At $ t = 0 $, the exponent term equates to 1. As such, the following are true:
\begin{itemize}
  \item $ q = 0 $
  \item $ i = \frac{\mathcal{E}}{R} $
  \item $ V_C = 0 $
  \item The capacitor acts as a broken wire, providing no electric potential.
\end{itemize}
As $ t \to \infty $, the exponent term tends to 0. As such, the following are true:
\begin{itemize}
  \item $ q \to C \mathcal{E} $
  \item $ i \to 0 $
  \item $ V_C \to \mathcal{E} $
  \item The capacitor acts as a zero-resistance normal wire, providing the full amount of electric potential.
\end{itemize}

\subsection{Discharging a Capacitor}
Again, the intricacies of deriving these equations isn't needed in this course. When discharging a capacitor:
\[ q = q_0 e^{-t/(RC)} \]
\[ i = -\left(\frac{q_0}{RC}\right) e^{-t/(RC)} \]
Think about that happens when $ t = 0 $ and as $ t \to \infty $.

\section{One Final Aside}
The $ RC $ constant in the exponent has a special name and symbol, the \emph{capacitive time constant}.
\[ \tau_C = RC \]

\end{document}
