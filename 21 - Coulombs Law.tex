\documentclass[]{article}
\usepackage[T1]{fontenc}
\usepackage{amsmath}
\usepackage{siunitx}

\begin{document}
\section{The Basics}
\begin{itemize}
  \item Like charges repel, and opposite charges attract. Charge moves freely in a conductor like metal but stays fixed in an insulator.
  \item You can charge by induction, which is when both a ground and a charge are touching a conductor, then you move both away quickly.
  \item Charge is quantized. The charge of a proton is $ e = $ \qty{1.602e-19}{\coulomb}
  \item Charge is conserved. The total amount (hear charge) in a system remains constant within the system.
  \item If objects are identical and conductive, their charge averages.
    \begin{itemize}
      \item e.g. Touching a 3C charge to a 6C charge makes both of their charges 4.5C.
      \item Touching a charged object to ground zeroes the charge.
    \end{itemize}
\end{itemize}

\section{The Formula}
\[ \vec{F} = k \frac{q_1 q_2}{r^2} \hat{r} \]
where $ k = \frac{1}{4 \pi \epsilon_0} $ and $ \epsilon_0 = $ \qty{8.85e-12}{\coulomb\squared\per\newton\meter\squared}

For most purposes, it is recommended to use Coulomb's Law for magnitude, then reference diagrams for direction.

\section{Shell Theorems}
Outside a uniformly distributed charged shell, the force is the same as for a point particle at the center of the shell. Inside the shell, the force is 0.
\end{document}
