\documentclass[]{article}
\usepackage[T1]{fontenc}
\usepackage{amsmath}
\usepackage{siunitx}

\begin{document}
\section{The Basics}
\begin{itemize}
  \item Electric field exists everywhere in space.
  \item The direction of the field is based on the force enacted on a positive charge
  \item The field (represented with "field lines") goes away from positive charges and towards negative charges
\end{itemize}
\subsection{Superposition Principle}
You can add the electric fields at any given point to get the "net" electric field.
\section{General Formulas}
In a completely general case, the electric field $ E $ is defined as the following:
\[ \vec{E} = \frac{\vec{F}}{q_{0}} \]
For point charges, the force enacted on each other is the same as Coulomb's Law, just with different labels. This results in the following:
\[ |E| = \frac{1}{4\pi\epsilon_0}\frac{q}{r^2} \]
\section{Special Cases}
A single equation for a point charge is not really useful. As such, we can derive equations for other cases by using $ \lambda = \frac{q}{\ell} $ as the charge per unit length and $ \sigma = \frac{q}{A} $ as the charge per unit area.
We derive equations by using the derivative of $ q $ expressed with $ \lambda \,ds $ or $ \sigma dA $. From there, we can integrate the derivative of the point equation for $ E $ (which is the exact same as the normal equation, just with $ dq $ instead of $ q $) to get the true equation for $ E $.
\subsection{Ring of Charge (electric field on centered axis)}
Since we're dealing with a ring, $ \ell = 2 \pi R $, where $ R $ is the radius of the ring.
\[ \lambda = \frac{q}{2 \pi R} \Rightarrow dq = \lambda \,ds \]
After integrating $ dE $ (see pg 640 for details), we arrive at the following equation for an electric field $ z $ units above the axis of a ring:
\[ |E_z| = \frac{1}{4 \pi \epsilon_0} \frac{q z}{(z^2 + R^2)^{3/2}} \]
Note: All electric field in the x and z directions cancel out, hence only the $ E_z $.
\subsection{Arc of Charge (electric field in plane)}
Assuming a perfect 1/4 circle around the origin, $ |E_x| = |E_y| $.
\[ |E_x| = \frac{1}{4\pi\epsilon_0} \frac{\lambda}{r} \]
\subsection{Line Segment of Charge (field \emph{parallel} to wire)}
\[ |E_x| = \frac{\lambda}{4\pi\epsilon_0}\left(\frac{-1}{r_2} + \frac{1}{r_1}\right) \]
where $ r_1 $ is the distance between the point and the wire and $ r_2 $ is $ r_1 + \ell $.
\subsection{Disk of Charge (electric field on centered axis)}
\[ E_z = \frac{\sigma}{2\epsilon_0}\left(1 - \frac{z}{\sqrt{z^2 + R^2}}\right) \]
\subsubsection{A Curious Note}
\[ \lim_{R \to \infty} E_z = \frac{\sigma}{2\epsilon_0} \]
which is the equation for the electric field at a point P on an axis normal to a plane.

\section{Dipole in an Electric Field}
When a dipole is placed in a uniform electric field, it experiences a \emph{torque} due to the electric field wanting to oppose the charges in different directions.
For a dipole with both ends having equal and opposite charges, separated by a distance $ d $, and in an electric field $ \vec{E} $,
\[ \vec{\tau} = \vec{p} \times \vec{E} \]
\[ U = -\vec{p} \cdot \vec{E} \]
where $ p = qd $ is the moment on a dipole. Additionally, the following holds:
\[ |\vec{\tau}| = |\vec{p} \times \vec{E}| = pE\sin(\theta) \]
\end{document}
